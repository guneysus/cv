
\cvsection{Tecrübeler}
\begin{cventries}

    \cventry
        {Backend Developer}
        {Neyasis Teknoloji}
        {İstanbul}
        {Eylül 2015 - Ekim 2015}
        {
        \begin{cvitems}
            \item{Entity Framework, ASP.Net MVC5 ile standart bir web uygulaması geliştirdim. }
        \end{cvitems}
        }
        
        \cventry
        {Web Developer}
        {Doğan TV Holding A.Ş.}
        {İstanbul}
        {Ekim 2015 - Mart 2016}
        {
            \begin{cvitems}
                \item { TEVE2 (eski adıyla TV2) web sitesi frontend geliştirmelerde bulundum. }
                \item {\href{https://play.google.com/store/apps/details?id=dogantv.ekranda}{Ekranda} }
                { mobil uygulaması için Restful API. Tornado, MongoDB, AWS RDS}
                \item { Selenium e2e yardımıyla, TEVE2 web sitesinin reklamlarını test eden bir dizi test yazdım. }
                \item { Python Celery (Distributed Task Queue) yardımıyla ile video encoding, YouTube/SFTP video yükleme ve çeşitli işlerin takibi için Angular ile web uygulaması yazdım. }
            \end{cvitems}
        }
        
        \cventry{Software Developer}{NSC Teknoloji}{İstanbul}{Nisan 2016 - Ağustos 2016}
        {
            \begin{cvitems}
                \item{ C\# ve MSSQL (SQL Server Agent) ile Excel raporları oluşturan Windows Servisleri geliştirip bakımlarını yaptım. }
            \end{cvitems}
        }
        \cventry{Web Developer}{Ekin Teknoloji}{İstanbul}{Kasım 2016 - Ekim 2017 }
        {
            \begin{cvitems}
                \item { Arama işleri için MSSQL Elastic Search dönüşümünü yaptım. }
                \item { ASP.Net SignalR ile gerçek zamanlı iletişim sistemi ve simülasyonunun geliştirdim.}
                \item { WCF ASP.Net Core WebApi dönüşümünü yapıp unit testlerini yazdım. }
                \item { Websocket ile çalışan bir C++ uygulamasına, ASP.Net Core WebApi ve Vue.js ile arayüz geliştirdim ve Ubuntu üzerine Nginx ile yayınladım. }
            \end{cvitems}
        }

        \cventry{Fullstack Developer}{Bilge Adam Güney Ekibi - Genel Merkez}{İstanbul}{Kasım 2017 - Mart 2019 }
        {
            \begin{cvitems}
                \item { Bir e-Ticaret uygulamasının geliştirilmesinde başından yayına alınmasına kadar fullstack olarak  görev aldım. }
                \item { ASP.Net Core ile ürün görsellerini resize eden bir ImageServis geliştirdim ve AWS CloudFront CDN hizmetini kullandım. }
                \item { Geliştirme ve test ortamımız olan AWS EC2 üzerindeki Window Server makinesi üzerinde remote debugging yöntemleriyle hata tespitlerinde bulundum. }
                \item { "AWS IoT Hizmetleri" ve ".NET Core Yazılımcılar İçin Linux - 101" ve ".NET Core Yazılımcılar İçin Linux - 102" sunumları yaptım. }
            \end{cvitems}
        }

        \cventry{Software Developer}{General Mobile}{İstanbul}{Mart 2019 - Ekim 2019}
        {
            \begin{cvitems}
                \item { ... }
            \end{cvitems}
        }
        
        \cventry{Software Developer}{adesso Turkey/Architecht Bilişim}{İstanbul}{Kasım 2019 - }
        {
            \begin{cvitems}
                \item { ... }
            \end{cvitems}
        }        
        
\end{cventries}
