
\cvsection{Tecrübeler}
\begin{cventries}

    \cventry
        {Backend Developer}
        {Neyasis Teknoloji}
        {İstanbul}
        {Eylül 2015 - Ekim 2015}
        {
        \begin{cvitems}
            \item{Entity Framework, ASP.Net MVC5 ile bir web uygulamasının kullanıcı kayıt, şifre sıfırlama ve Excel raporlarının oluşturulması. }
        \end{cvitems}
        }
        
        \cventry
        {Web Developer}
        {Doğan TV Holding A.Ş.}
        {İstanbul}
        {Ekim 2015 - Mart 2016}
        {
            \begin{cvitems}
                \item{TEVE2 (eski adıyla TV2) web sitesi. ASP.Net MVC}
                \item {\href{https://play.google.com/store/apps/details?id=dogantv.ekranda}{Ekranda} }
                { mobil uygulaması için Restful API. Tornado, MongoDB, AWS RDS}
                \item {Selenium ile TEVE2 web sitesinin reklamlarının testi}
                \item {Video encoding, YouTube/SFTP video yükleme ve çeşitli işler için Celery, Flower ve Angular ile web uygulaması.}
            \end{cvitems}
        }
        
        \cventry{Software Developer}{NSC Teknoloji}{İstanbul}{Nisan 2016 - Ağustos 2016}
        {
            \begin{cvitems}
                \item{C\# ve MSSQL (SQL Server Agent) ile saha bilgisayarlarında çalışan windows servislerinin bakımı ve geliştirilmesi}
            \end{cvitems}
        }
        \cventry{Web Developer}{Ekin Teknoloji}{İstanbul}{Kasım 2016 - Ekim 2017 }
        {
            \begin{cvitems}
                \item { Elastic Search entegrasyonunun yapılması ve operasyon ekibine eğitim verilmesi}
                % \item { Çeşitli windows servislerinin ve data erişim kütüphanelerinin geliştirilmesi.}
                \item { Continuous Delivery ile (yerel NuGet sunucusu) verimliliğin artırılması, TFS üzerinde konfigürasyonların tanımlanması}
                \item { ASP.Net SignalR ile gerçek zamanlı iletişim sistemi ve simülasyonunun geliştirilmesi,}
                % \item { Mevcut WCF servisinin refactor edilerek ASP.Net WebApi ile geliştirilmesi ve testlerinin yazılması}
                % \item { Çeşitli yardımcı kütüphane ve araçların yazılması.}
                \item { ASP.Net Core, Vue.js Restful API ile web uygulaması geliştirilmesi. Ubuntu üzerine Nginx ile yayınlanması}
            \end{cvitems}
        }

        \cventry{Fullstack Developer}{Bilge Adam Güney Ekibi - Genel Merkez}{İstanbul}{Kasım 2017 - }
        {
            \begin{cvitems}
                \item { }
            \end{cvitems}
        }
        
\end{cventries}
